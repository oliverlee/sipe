\documentclass[11pt,a4paper]{article}
% See geometry.pdf to learn the layout options. There are lots.
\usepackage{geometry}
\geometry{left = 3cm, right = 3cm, top = 3cm, bottom = 3cm}
% Activate to begin paragraphs with an empty line rather than an indent
\usepackage[parfill]{parskip}
\usepackage[table]{xcolor}
\usepackage{courier,multirow,fancyhdr}
\usepackage{float,amsmath,graphicx,framed,subfiles}
\usepackage{amssymb,subcaption,textcomp,listings}
\usepackage{booktabs,epstopdf,caption,units}
% hyperref should be loaded last
\usepackage[
    colorlinks=true,
    pdfencoding=auto,
    pdfauthor={Oliver Lee},
    pdftitle={WB2301-5 System Identification and Parameter Estimation:
    Assignment 3 - Identification},
    pdftex
]{hyperref}

\lstset{
language=Matlab,
basicstyle=\footnotesize\ttfamily,
showspaces=false,
showstringspaces=false,
showtabs=false,
frame=single,
tabsize=2,
captionpos=b,
breaklines=false,
breakatwhitespace=false,
numbers=left,
numbersep=4pt,
}

\newcommand{\norm}[1]{\left\lVert#1\right\rVert}

\frenchspacing

\pagestyle{fancy}
\fancyhead{}
\fancyhead[R]{Lee}
\fancyhead[L]{Assignment 3}


\title{WB2301-5 System Identification and Parameter Estimation \\
Assignment 3 - Identification}
\author{Oliver Lee}
\date{\today}
\graphicspath{ {images/} }

\begin{document}
\maketitle

\section{Time Domain Models in Closed Loop}

% 1a
\subsection{Derive Correct Model Structure}
Given a general model structure of:
\begin{equation}
    y(t) = G(q)u(t) + H(q)e(t) \label{eq:gen_model}
\end{equation}
where $y(t)$ is the output signal, $u(t)$ is the input signal, $e(t)$ is noise,
and $H(q)$ represents the noise dynamics and $G(q)$ represents the system
dynamics. From the system, we can express $y(t)$ in terms of $n(t)$ and $u(t)$:
\begin{equation}
    y(t)= H(q)u(t) + n(t) \label{eq:y1}
\end{equation}
It can be shown that the expressions for $u(t)$ and $y(t)$ are
given by the following expressions:
\begin{align}
    u(t) &= \frac{1}{1 + H(q)G(q)}r(t) -
        \frac{G(q)}{1 + H(q)G(q)}n(t) \label{eq:u} \\
    y(t) &= \frac{H(q)}{1 + H(q)G(q)}r(t) +
        \frac{1}{1 + H(q)G(q)}n(t) \label{eq:y2}
\end{align}

\subsubsection{Two Stage Method}
\label{sec:1a1}
For the Two Stage Method, we need the determine models for a first stage from
$r(t)$ to $u(t)$ and a second stage from $u^\prime(t)$ to $y(t)$. For stage 1,
we start with equation (\ref{eq:u}) and split $H$ and $G$ into numerator and
denominator terms:
\begin{align}
    u(t) &= \frac{1}{1 + \frac{H_nG_n}{H_dG_d}}r(t) +
        \frac{-\frac{G_n}{G_d}}{1 + \frac{H_nG_n}{H_dG_d}}n(t) \notag \\
    &= \frac{H_dG_d}{H_dG_d + H_nG_n}r(t) +
        \frac{-H_dG_n}{H_dG_d + H_nG_n}n(t)  \label{eq:model11}
\end{align}
Note that the argument of the $G(q)$ and $H(q)$ systems is omitted. As the
denominators of the transfer functions from $r(t)$ to $u(t)$ and from $n(t)$ to
$u(t)$ are identical while the numerators differ, we can write an expression
for this stage as:
\begin{equation}
    u(t) = \frac{B(q)}{A(q)}r(t) + \frac{C(q)}{A(q)}n(t) \label{eq:armax}
\end{equation}
which corresponds to an ARMAX model structure.

For stage 2, if we assume `noise free' input $u^\prime(t)$ to $y(t)$, we have
the equation:
\begin{equation}
    y(t) = \frac{H_n}{H_d}u^\prime(t) + n(t) \label{eq:model12}
\end{equation}
and can be rewritten as:
\begin{equation}
    u(t) = \frac{B(q)}{F(q)}r(t) + n(t) \label{eq:oe}
\end{equation}
which corresponds to an Output Error (OE) model struture.

\subsubsection{Coprime Method}
For the Coprime Method, we need the determine models for a first stage from
$r(t)$ to $u(t)$ and a second stage from $r(t)$ to $y(t)$. For stage 1, we can
examine equation (\ref{eq:u}) and we see that we have the same form as the
first stage in the Two Stage method (\ref{eq:model11}), resulting in an ARMAX
model structure

For stage 2, we examine equation (\ref{eq:y2}) and we see that:
\begin{align}
    y(t) &= \frac{\frac{H_n}{H_d}}{1 + \frac{H_nG_n}{H_dG_d}}r(t) +
        \frac{1}{1 + \frac{H_nG_n}{H_dG_d}}n(t) \notag \\
    &= \frac{G_dH_n}{H_dG_d + H_nG_n}r(t) +
        \frac{H_dG_d}{H_dG_d + H_nG_n}n(t)  \label{eq:model22}
\end{align}
this also matches the ARMAX model equation (\ref{eq:armax}).

% 1b
\subsection{Compare Estimated Frequency Response Functions}
\begin{figure}
    \begin{center}
        \includegraphics[width=\textwidth]{question1b.eps}
    \end{center}
    \caption{Frequency spectrum of $H(f)$, $\hat{H}(f)$}
    \label{fig:1b}
\end{figure}
The FRF of system $H(f)$ is plotted in figure \ref{fig:1b}. The two stage
estimate is determined by using the ARMAX model of the first stage to generate
a new `noise free' signal $u^\prime(t)$. Signals $u^\prime(t)$ and $y(t)$ are
then fit to an OE model which we take to be $\hat{H}_{2stage}(f)$. The coprime
estimate is determined by first fitting ARMAX models for both stages of the
system. We can represent the two ARMAX models as:
\begin{equation*}
    U/R = \frac{1}{1 + HG} \qquad Y/R = \frac{H}{1 + HG}
\end{equation*}
resulting in the estimate $\hat{H}_{coprime}(f) = \frac{Y/R}{U/R}$.
FIXME: discuss goodness of estimates

% 1c
\subsection{Disadvantages of Time Domain Model Estimators}
\subsubsection{Two Stage Method}
FIXME: The two stage method only works for stable subsystems.

\subsubsection{Coprime Method}
FIXME: The denominator of the two estimates should be equal and cancel, but are
never exactly equal for collected data and the result is a higher order model.

\end{document}
