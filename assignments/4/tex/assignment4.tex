\documentclass[11pt,a4paper]{article}
% See geometry.pdf to learn the layout options. There are lots.
\usepackage{geometry}
\geometry{left = 3cm, right = 3cm, top = 3cm, bottom = 3cm}
% Activate to begin paragraphs with an empty line rather than an indent
\usepackage[parfill]{parskip}
\usepackage[table]{xcolor}
\usepackage{courier,multirow,fancyhdr}
\usepackage{float,amsmath,graphicx,framed,subfiles}
\usepackage{amssymb,subcaption,textcomp,listings}
\usepackage{booktabs,epstopdf,caption,units}
\usepackage{tabularx}
\usepackage[framed,numbered]{matlab-prettifier}
\usepackage[T1]{fontenc}
\usepackage{lmodern}
% hyperref should be loaded last
\usepackage[
    colorlinks=true,
    pdfencoding=auto,
    pdfauthor={Oliver Lee},
    pdftitle={WB2301-5 System Identification and Parameter Estimation:
    Assignment 4 - Optimization & Real Data Analysis},
    pdftex
]{hyperref}

\newcommand{\norm}[1]{\left\lVert#1\right\rVert}
\newcommand{\mcode}[1]{
    \lstinline[
        style=Matlab-editor,
        basicstyle=\mlttfamily,
    ]{#1}
}

\frenchspacing

\pagestyle{fancy}
\fancyhead{}
\fancyhead[R]{Lee}
\fancyhead[L]{Assignment 4}


\title{WB2301-5 System Identification and Parameter Estimation \\
Assignment 4 - Optimization \& Real Data Analysis}
\author{Oliver Lee}
\date{\today}
\graphicspath{ {images/} }

\begin{document}
\maketitle

\section{Optimization Techniques}

% 1a
\subsection{Optimization with Grid and Gradient Search}
Using a model $ \hat{y}(t) $ assumed to be:
\begin{equation*}
    \hat{y}(t) = a\cos(\frac{bt}{2}) + b\sin(\frac{at}{2})
\end{equation*}
and parameters $ a,\: b \in [0, 10] $. To find the optimal values of $a$ and $b$,
the error function $e$ to minimize is defined as:
\begin{equation*}
    e = (y(t) - \hat{y}(t))^T (y(t) - \hat{y}(t))
\end{equation*}
where $y(t),\: \hat{y}(t) \in \mathbb{R}^{n \times 1} $ and $n$ is the number
of elements of the considered time vector.

\autoref{fig:optsurf} shows the error calculation for parameters $a, b$ with a
resolution of $0.1$ (a 2-d grid) as well as multiple pathways from the use of
gradient search with different initial conditions. All initial conditions are
chosen randomly with a uniform distribution from the search space, except for
the given condition $a = 5, \: b = 5$. As can be seen, poor choices in initial
conditions result in convergence at local minima not equal to the global
minimum. In fact, for the given problem and the randomly chosen initial
conditions, most do not converge at the global minimum.  If the error function
surface (or hypersurface for a general optimization problem) is not known a
priori (i.e. grid search was not performed before use of gradient search), it
is possible to execute a number of gradient searches which only return a local
minimum without knowing that the global minimum exists elsewhere in the search
space.

\begin{figure}
    \centering
    \includegraphics[width=\textwidth]{optsurf.eps}
    \caption{Error Function Surface with Gradient Search Pathways}
    \label{fig:optsurf}
\end{figure}

% 1b
\subsection{Advantages and Disadvantages of Various Optimization Techniques}
\begin{table}
    \centering
    \begin{tabular}{|r|r|r|r|r|}
        \hline
%        \multicolumn{5}{|c|}{Performance of Optimization Techniques} \\
%        \hline
        \nonumber & \shortstack[c]{estimated\\parameters (a, b)} &
            \shortstack[c]{error\\residual} &
            \shortstack[c]{number of\\iterations} &
            \shortstack[c]{calculation\\time} \\
        \hline
        Grid Search & $ (5.8,\: 3.5) $ & $ 8.4835 $ & $ 10201 $ &
            $ 0.9455 $ s \\
        Gradient Search & $ (5.97,\: 3.53) $ & $ 5.3066 \times 10^{-17} $ &
            $ 8 $ & $ 0.6253 $ s \\
        Genetic Algorithm & $ (5.97,\: 3.53) $ & $ 2.2417 \times 10^{-7} $ &
            $ 100 $ & $ 4.4695 $ s \\
        \hline
    \end{tabular}
    \caption{Performance of Optimization Techniques}
    \label{tab:optperf}
\end{table}
In \autoref{tab:optperf} gradient search uses an initial condition of $ (5,\:
5) $. Performance of gradient search with different initial conditions will
result in small variations of estimated parameters, error residual, number of
iterations, and calculation time unless the initial condition is poorly choosen
and the search finds a different local minimum. Perfomance in the genetic
algorithm (GA) will also vary slightly in difference instances due to
randomness associated with genetic operations although the perfomance values
can be expected to be of the same order. Refer to \autoref{fig:optsurf} for
examples of poorly chosen initial conditions resulting in a different local
minimum with gradient search. \autoref{tab:optperf} shows that grid search
results in a large number of searches and does find the global minimum,
although the precision of the estimated parameters is limited by the resolution
of the grid used. Both gradient search and the genetic algorithm find the same
minimum, with higher precision. Gradient search is able to do quickly, with
fewest function iterations and lowest computation time. The GA is also able to
find the global minimum is fewer function iterations than grid search (orders
of magnitude lower) but total computation time is higher due to creation of
candidate solutions and selection/genetic operations used in creating
subsequent generations. Note that the number of iterations for GA is the number
of generations; the number of function evaluations is equal to product of
number of iterations and population size. For this problem, the population size
is $50$ giving a total number of function evaluations of $5000$. Although the
GA optimization takes longer in this
problem, larger and more complex optimization problems may have jacobian and
hessian matrices that are more expensive to compute (in time and memory) for
gradient based methods and GA may faster in some cases.

Advantages and disadvantages for the three optimization techniques can be
summarized below:
\begin{itemize}
    \item Grid Search
        \begin{itemize}
            \item Will determine global minimum (with given grid resolution)
            \item Astronomical number of function evalutaions (dependent on
                grid resolution)
        \end{itemize}
    \item Gradient Search
        \begin{itemize}
            \item Very fast convergence near optimum
            \item Dependent on initial conditions
        \end{itemize}
    \item Genetic Algorithm
        \begin{itemize}
            \item Does not need to compute function gradients
            \item Slow convergence near optimum
        \end{itemize}
\end{itemize}

% 1c
\subsection{``Improvements'' for Grid Search}
The estimate for grid search can be improved by increasing the resolution of
the search grid, that is, increasing the resolution for the candidate solutions
of $a$ and $b$. While this is simple to implement, a 10x increase in resolution
for both $a$ and $b$ increases the number of function evaluations by a factor
of 100. This is expensive computationally, although grid search is easily
parallized and may be a feasible solution if resources are available.

% 1d
\subsection{Options of Genetic Algorithms}
Genetic Algorithms have a number of options that affect performance of
optimization.
\begin{itemize}
    \item\mcode{PopulationSize}- Size of the population. A population size
        that is too large can result in slow convergence due to computational
        cost associated with large amounts of data. A size that is too small
        can also result in slow convergence due to limited search range as a
        result of too few genetic samples. If elitist selection and converge is
        poorly tuned, a small population size can also result in premature
        convergence.
    \item\mcode{EliteCount}- The positive integer specifying how many
        individuals survive in the next generation. With zero parent solutions
        surviving in subsequent generations, there may be a loss of good
        solutions. With too many, the ability to search the rest of the
        parameter space is reduced resulting more generations necessary for
        convergence or convergence in a local minimum.
    \item\mcode{CrossoverFraction}- The fraction of the population of
        subsequent generations created by crossover of parent solutions from
        the previous generation. A fraction that is too large can lead to
        premature convergence; the amount of children selected from mutation
        will not be large enough to explore other areas of the parameter space
        before convergence in a local minimum due to crossover/elitist
        selection. A fraction that is too small can lead to low genetic
        variation, resulting to slow convergence as a result of high mutation
        rate.
    \item\mcode{PopInitRange}- The range of parameter values for the initial
        population. A poorly formed initial range can result in an increase in
        generations required for convergence as more time is necessary to span
        the search space through crossover and mutation. If information
        regarding areas where optimal solutions exist, the PopInitRange can be
        formuated so that the initial population begins in those areas.
    \item\mcode{Generations}- The maximum number of iterations before the
        algorithm halts. If chosen too small, the algorithm may halt before
        convergence.  The algorithm stops wonce convergence is reached, however
        poor fomulation for convergence could lead to unnecessary and time
        consuming iterations.
\end{itemize}

% 1e
\subsection{Combination of Grid and Gradient Search}
Grid and gradient search can be combined to robustly by reducing the resolution
of grid search and using each point as an initial condition for gradient
search. As gradient search has fast convergence, the increase in function
evaluations due to gradient search is smaller than the decrease in function
evaluations due to the reduction in grid resolution. With the number of
iterations of each gradient search on the order of 10 and the decrease of
resolution by 10 for each $a$ and $b$, the number of function evaluations of
the combined search is an order of magnitude smaller than grid search alone.

Using the search function decribed in \autoref{lst:combsearch} to optimize
parameters $a$ and $b$, we find the performance metrics as described in
\autoref{tab:combperf}. The pathways of the gradient searches are shown in
\autoref{fig:combsearch} using a resolution of $1$ for $a$ and $b$ for the grid
resolution. While the global minimum is found, we can see visually that some of
the pathways converge at local minima and not the global minimum and show the
sensitivity of gradient search to the initial condition. While total
computation time is quite large compared to both grid search and a single
instance of gradient search (see \autoref{tab:optperf}), it is comparable to
GA. Number of iterations is the sum of the gradient search iterations for all
initial conditions and is equal to the total number of function evaluations. As
expected, the number of iterations is an order of magnitude less than simple
grid search. The parameter estimates match the values for gradient search (with
a good initial condition) and GA.

\newpage
\lstset{
    caption=Combined Grid and Gradient Search Function,
    captionpos=b,
    label={lst:combsearch},
}
\lstinputlisting[
    style=Matlab-editor,
    basicstyle=\footnotesize\mlttfamily,
]{../combinedsearch.m}

\begin{table}
    \centering
    \begin{tabular}{|r|r|r|r|r|}
        \hline
%        \multicolumn{5}{|c|}{Performance of Optimization Techniques} \\
%        \hline
        \nonumber & \shortstack[c]{estimated\\parameters (a, b)} &
            \shortstack[c]{error\\residual} &
            \shortstack[c]{number of\\iterations} &
            \shortstack[c]{calculation\\time} \\
        \hline
        Combined Search & $ (5.97,\: 3.53) $ & $ 1.8310 \times 10^{-28} $ &
            $ 1948 $ & $ 4.2724 $ s \\
        \hline
    \end{tabular}
    \caption{Performance of Combined Grid and Gradient Search}
    \label{tab:combperf}
\end{table}

\begin{figure}
    \centering
    \includegraphics[width=\textwidth]{combsearch.eps}
    \caption{Gradient Search Pathways from Combined Search}
    \label{fig:combsearch}
\end{figure}



\section{Real Data Analysis}
% FIXME: What movie? WriteExperiment.3GP is of low resolution and you can't see
% the monitor.

% 2a
\subsection{Figure 1?}

% 2b
\subsection{Coherence of Whole Dataset}

% 2c
\subsection{Validity of LTI identification}

% 2d
\subsection{Dynamic Characteristics and Parameter Estimation}
A mass-spring-damper system affects the frequency response most significantly
near the natural frequency of the system $\omega_n$, which is given by:
\begin{equation}
    \omega_n = \sqrt{\frac{k}{m}}
\end{equation}
Note that the damping of the system does not affect the natural frequency of
the system.

% 2e
\subsection{Estimation of Mass, Spring, Damper Parameters}

% 2f
\subsection{Parameter Estimation with Frequency and Coherence Weighting}

% 2g
\subsection{Derivation of Transfer Function with Velocity Feedback}

% 2h
\subsection{Parameter Estimation of Transfer Function with Velocity Feedback}

% 2i
\subsection{SEM of Estimated Parameters}

% 2j
\subsection{Parameters Dependent on Background Torque}

% 2k
\subsection{VAF of Segments}

% 2l
\subsection{Use of High-Pass Filters}


\end{document}
