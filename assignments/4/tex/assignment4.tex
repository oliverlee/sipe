\documentclass[11pt,a4paper]{article}
% See geometry.pdf to learn the layout options. There are lots.
\usepackage{geometry}
\geometry{left = 3cm, right = 3cm, top = 3cm, bottom = 3cm}
% Activate to begin paragraphs with an empty line rather than an indent
\usepackage[parfill]{parskip}
\usepackage[table]{xcolor}
\usepackage{courier,multirow,fancyhdr}
\usepackage{float,amsmath,graphicx,framed,subfiles}
\usepackage{amssymb,subcaption,textcomp,listings}
\usepackage{booktabs,epstopdf,caption,units}
% hyperref should be loaded last
\usepackage[
    colorlinks=true,
    pdfencoding=auto,
    pdfauthor={Oliver Lee},
    pdftitle={WB2301-5 System Identification and Parameter Estimation:
    Assignment 4 - Optimization & Real Data Analysis},
    pdftex
]{hyperref}

\lstset{
language=Matlab,
basicstyle=\footnotesize\ttfamily,
showspaces=false,
showstringspaces=false,
showtabs=false,
frame=single,
tabsize=2,
captionpos=b,
breaklines=false,
breakatwhitespace=false,
numbers=left,
numbersep=4pt,
}

\newcommand{\norm}[1]{\left\lVert#1\right\rVert}

\frenchspacing

\pagestyle{fancy}
\fancyhead{}
\fancyhead[R]{Lee}
\fancyhead[L]{Assignment 4}


\title{WB2301-5 System Identification and Parameter Estimation \\
Assignment 4 - Optimization \& Real Data Analysis}
\author{Oliver Lee}
\date{\today}
\graphicspath{ {images/} }

\begin{document}
\maketitle

\section{Optimization Techniques}

% 1a
\subsection{Optimization with Grid and Gradient Search}
% Given a general model structure of:
% \begin{equation}
%     y(t) = G(q)u(t) + H(q)e(t) \label{eq:gen_model}
% \end{equation}
% where $y(t)$ is the output signal, $u(t)$ is the input signal, $e(t)$ is noise,
% and $H(q)$ represents the noise dynamics and $G(q)$ represents the system
% dynamics. From the system, we can express $y(t)$ in terms of $n(t)$ and $u(t)$:
% \begin{equation}
%     y(t)= H(q)u(t) + n(t) \label{eq:y1}
% \end{equation}
% It can be shown that the expressions for $u(t)$ and $y(t)$ are
% given by the following expressions:
% \begin{align}
%     u(t) &= \frac{1}{1 + H(q)G(q)}r(t) -
%         \frac{G(q)}{1 + H(q)G(q)}n(t) \label{eq:u} \\
%     y(t) &= \frac{H(q)}{1 + H(q)G(q)}r(t) +
%         \frac{1}{1 + H(q)G(q)}n(t) \label{eq:y2}
% \end{align}

% 1b
% \subsection{Advantages and Disadvantages of Various Optimization Techniques}
% \begin{figure}
%     \begin{center}
%         \includegraphics[width=\textwidth]{question1b.eps}
%     \end{center}
%     \caption{Frequency spectrum of $H(f)$, $\hat{H}(f)$}
%     \label{fig:1b}
% \end{figure}
% The FRF of system $H(f)$ is plotted in figure \ref{fig:1b}. The two stage
% estimate is determined by using the ARMAX model of the first stage to generate
% a new `noise free' signal $u^\prime(t)$. Signals $u^\prime(t)$ and $y(t)$ are
% then fit to an OE model which we take to be $\hat{H}_{2stage}(f)$. The coprime
% estimate is determined by first fitting ARMAX models for both stages of the
% system. We can represent the two ARMAX models as:
% \begin{equation*}
%     U/R = \frac{1}{1 + HG} \qquad Y/R = \frac{H}{1 + HG}
% \end{equation*}
% resulting in the estimate $\hat{H}_{coprime}(f) = \frac{Y/R}{U/R}$.

% 1c
\subsection{``Improvements'' for Grid Search}

% 1d
\subsection{Parameters of Genetic  Algorithms}

% 1e
\subsection{Combination of Grid and Gradient Search}
% FIXME: How is a "function evaluation" defined? Is it a single evaluation
% of the output? What about calculation of the Jacobian and Hessian?

\section{Real Data Analysis}
% FIXME: What movie? WriteExperiment.3GP is of low resolution and you can't see
% the monitor.

% 2a
\subsection{Figure 1?}

% 2b
\subsection{Coherence of Whole Dataset}

% 2c
\subsection{Validity of LTI identification}

% 2d
\subsection{Dynamic Characteristics and Parameter Estimation}
A mass-spring-damper system affects the frequency response most significantly
near the natural frequency of the system $\omega_n$, which is given by:
\begin{equation}
    \omega_n = \sqrt{\frac{k}{m}}
\end{equation}
Note that the damping of the system does not affect the natural frequency of
the system.

% 2e
\subsection{Estimation of Mass, Spring, Damper Parameters}

% 2f
\subsection{Parameter Estimation with Frequency and Coherence Weighting}

% 2g
\subsection{Derivation of Transfer Function with Velocity Feedback}

% 2h
\subsection{Parameter Estimation of Transfer Function with Velocity Feedback}

% 2i
\subsection{SEM of Estimated Parameters}

% 2j
\subsection{Parameters Dependent on Background Torque}

% 2k
\subsection{VAF of Segments}

% 2l
\subsection{Use of High-Pass Filters}


\end{document}
