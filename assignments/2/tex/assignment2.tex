\documentclass[11pt,a4paper]{article}
% See geometry.pdf to learn the layout options. There are lots.
\usepackage{geometry}
\geometry{left = 3cm, right = 3cm, top = 3cm, bottom = 3cm}
% Activate to begin paragraphs with an empty line rather than an indent
\usepackage[parfill]{parskip}
\usepackage{courier,multirow,fancyhdr}
\usepackage{float,amsmath,graphicx,color,framed,subfiles}
\usepackage{amssymb,subcaption,textcomp,listings}
\usepackage{booktabs,epstopdf,caption,units}
\usepackage[table]{xcolor}

\lstset{
language=Matlab,
basicstyle=\footnotesize\ttfamily,
showspaces=false,
showstringspaces=false,
showtabs=false,
frame=single,
tabsize=2,
captionpos=b,
breaklines=false,
breakatwhitespace=false,
numbers=left,
numbersep=4pt,
}
\newcommand{\norm}[1]{\left\lVert#1\right\rVert}
\frenchspacing

\pagestyle{fancy}
\fancyhead{}
\fancyhead[R]{Lee}
\fancyhead[L]{Assignment 2}

\title{WB2301-5 System Identification and Parameter Estimation \\
Assignment 2 - Open and Closed Loop Identification}
\author{Oliver Lee}
\date{\today}
\graphicspath{ {images/} }

\begin{document}
\maketitle

\section{Fourier Transformation of Signals}

% 1a
\subsection{Symmetric and Anti-symmetric Properties of Fourier Coefficients}
\begin{figure}
    \begin{center}
        \includegraphics[width=\textwidth]{question1a.eps}
    \end{center}
    \caption{Frequency spectrum of $y(t)$}
    \label{fig:1a}
\end{figure}
As all actual signals are of finite duration, there will always be Fourier
components at all frequencies. When these signals are sampled, any frequency
components above the Nyquist frequency (half the sampling frequency) will be
aliased, that is, components from higher frequencies will be indistinguishable
from components in lower frequencies.

For a real signal $x_0, \ldots, x_{N-1}$, with $N$ being the number of samples
of the signal, the discrete Fourier transform has the following property:
\begin{equation*}
    X_{N-k} = X_{-k} = X_{k}^{*}
\end{equation*}
where $X_{k}$ is the (complex) Fourier coefficient for frequency $k$ and
$X_{k}^{*}$ is the complex conjugate of $X_k$. If we look at the magnitude and
phase of $X_k$, we find that:
\begin{align*}
    \norm{X_{N-k}} = \norm{X_{k}^{*}} = \norm{X_{k}} \\
    \angle X_{N-k} = \angle X_{k}^{*}  = -\angle X_{k}
\end{align*}
Thus the magnitude of $X$ is symmetric about the Nyquist frequency and the
angle of $X$ is anti-symmetric about the Nyquist frequency. As a result, and as
commonly seen, the Fourier transform of a real signal is only plotted for
frequencies between 0 and the Nyquist frequency. The symmetric and
anti-symmetric properties can be seen in the magnitude and phase plots of the
Fourier transform of a signal $y(t)$ in figure \ref{fig:1a}.
%$u(t) = 2*sin(2\pi ft) + n(0, 1)$ which is a sum
%of a sinusoid with frequency of 3 Hz and amplitude 2 and white noise with mean
%0 and variance 1 in figure \ref{fig:1a}.

% 1b
\subsection{Effect of Sample Frequency}
\begin{figure}
    \begin{center}
        \includegraphics[width=\textwidth]{question1b.eps}
    \end{center}
    \caption{Magnitude spectrum of signal $y(t)$ sampled at different rates}
    \label{fig:1b}
\end{figure}
With higher sample frequency of a time signal, we have higher frequency
components in the Fourier transform. Note however, that the highest frequency
is not the sampling frequency, but the Nyquist frequency. Figure \ref{fig:1b}
shows the Fourier transforms for the same signal sampled at 4 different
frequencies. Each Fourier transform has been scaled by $1/N$, where $N$ is the
length of the signal in samples, such that the frequency components have the
same magnitudes across signals.

As our input signal is the sum of a sinusoid with white noise, we expect the
Fourier transform of the unfiltered signal to have a large component at \mbox{3
Hz} due to the sinusoidal signal and smaller, equal components at all other
frequencies due to noise. After a low-pass Butterworth filter is applied, the
frequency components above the cutoff frequency are attenuated by \mbox{20
dB/dec} per pole (or filter order). We can see in Figure \ref{fig:1b} that
attenuation of the noise begins around \mbox{25 Hz}, the cutoff frequency of
the applied filter, at a rate of \mbox{60 dB/dec} as the filter is third order.

% 1c
\subsection{Effect of Observation Time}
\begin{figure}
    \begin{center}
        \includegraphics[width=\textwidth]{question1c.eps}
    \end{center}
    \caption{Magnitude spectrum of signal $y(t)$ observed for different lengths
    of time}
    \label{fig:1c}
\end{figure}
A longer observation time $T$ of a time signal results in higher resolution in
the Frequency domain. In figure \ref{fig:1c}, a signal is sampled twice, once
for 10 seconds at \mbox{1000 Hz} and once for 5 seconds at \mbox{1000 Hz}, and
the resulting Fourier transforms are plotted. Note, the Fourier transforms in
figure \ref{fig:1c} have not been scaled by the length of the input signal or
any other value so that the magnitudes of the Fourier coefficients can be
compared in this case with equal sampling rate but differing observation time.

As the sampling frequency is the same, both Fourier transforms have frequency
components up to \mbox{1000 Hz}, but the signal with observation time $T = 10
s$, $y_{1000}$, has a frequency resolution of \mbox{0.1 Hz} while the signal
with observation time $T = 5 s$, $y_{1000, 5}$, has a frequency resolution of
\mbox{0.2 Hz}. While the components cannot be seen every 0.1 and \mbox{0.2 Hz}
in figure \ref{fig:1c}, the effect of a different frequency resolutions with a
noisy signal are visible. Since noise is present at all frequencies in
equivalent amounts, a lower sampling rate will result in larger frequency
components due to noise as the each FFT bin encompasses a large range of
frequencies. This can be seen by the higher noise floor in signal $y_{1000, 5}$
over signal $y_{1000}$.

\end{document}
