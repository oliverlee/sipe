\documentclass[11pt,a4paper]{article}
% See geometry.pdf to learn the layout options. There are lots.
\usepackage{geometry}
\geometry{left = 3cm, right = 3cm, top = 3cm, bottom = 3cm}
% Activate to begin paragraphs with an empty line rather than an indent
\usepackage[parfill]{parskip}
\usepackage[table]{xcolor}
\usepackage{courier,multirow,fancyhdr}
\usepackage{float,amsmath,graphicx,framed,subfiles}
\usepackage{amssymb,subcaption,textcomp,listings}
\usepackage{booktabs,epstopdf,caption,units}
% hyperref should be loaded last
\usepackage[
    colorlinks=true,
    pdfencoding=auto,
    pdfauthor={Oliver Lee},
    pdftitle={WB2301-5 System Identification and Parameter Estimation:
    Assignment 2 - Open and Closed Loop Identification},
    pdftex
]{hyperref}

\lstset{
language=Matlab,
basicstyle=\footnotesize\ttfamily,
showspaces=false,
showstringspaces=false,
showtabs=false,
frame=single,
tabsize=2,
captionpos=b,
breaklines=false,
breakatwhitespace=false,
numbers=left,
numbersep=4pt,
}

\newcommand{\norm}[1]{\left\lVert#1\right\rVert}

\frenchspacing

\pagestyle{fancy}
\fancyhead{}
\fancyhead[R]{Lee}
\fancyhead[L]{Assignment 2}


\title{WB2301-5 System Identification and Parameter Estimation \\
Assignment 2 - Open and Closed Loop Identification}
\author{Oliver Lee}
\date{\today}
\graphicspath{ {images/} }

\begin{document}
\maketitle

\section{Fourier Transformation of Signals}

% 1a
\subsection{Symmetric and Anti-symmetric Properties of Fourier Coefficients}
\begin{figure}
    \begin{center}
        \includegraphics[width=\textwidth]{question1a.eps}
    \end{center}
    \caption{Frequency spectrum of $y(t)$}
    \label{fig:1a}
\end{figure}
As all actual signals are of finite duration, there will always be Fourier
components at all frequencies. When these signals are sampled, any frequency
components above the Nyquist frequency (half the sampling frequency) will be
aliased, that is, components from higher frequencies will be indistinguishable
from components in lower frequencies.

For a real signal $x_0, \ldots, x_{N-1}$, with $N$ being the number of samples
of the signal, the discrete Fourier transform has the following property:
\begin{equation*}
    X_{N-k} = X_{-k} = X_{k}^{*}
\end{equation*}
where $X_{k}$ is the (complex) Fourier coefficient for frequency $k$ and
$X_{k}^{*}$ is the complex conjugate of $X_k$. If we look at the magnitude and
phase of $X_k$, we find that:
\begin{align*}
    \norm{X_{N-k}} = \norm{X_{k}^{*}} = \norm{X_{k}} \\
    \angle X_{N-k} = \angle X_{k}^{*}  = -\angle X_{k}
\end{align*}
Thus the magnitude of $X$ is symmetric about the Nyquist frequency and the
angle of $X$ is anti-symmetric about the Nyquist frequency. As a result, and as
commonly seen, the Fourier transform of a real signal is only plotted for
frequencies between 0 and the Nyquist frequency. The symmetric and
anti-symmetric properties can be seen in the magnitude and phase plots of the
Fourier transform of a signal $y(t)$ in figure \ref{fig:1a}.

% 1b
\subsection{Effect of Sample Frequency}
\begin{figure}
    \begin{center}
        \includegraphics[width=\textwidth]{question1b.eps}
    \end{center}
    \caption{Magnitude spectrum of signal $y(t)$ sampled at different rates}
    \label{fig:1b}
\end{figure}
With higher sample frequency of a time signal, we have higher frequency
components in the Fourier transform. Note however, that the highest frequency
is not the sampling frequency, but the Nyquist frequency. Figure \ref{fig:1b}
shows the Fourier transforms for the same signal sampled at 4 different
frequencies. Each Fourier transform has been scaled by $1/N$, where $N$ is the
length of the signal in samples, such that the frequency components have the
same magnitudes across signals.

As our input signal is the sum of a sinusoid with white noise, we expect the
Fourier transform of the unfiltered signal to have a large component at \mbox{3
Hz} due to the sinusoidal signal and smaller, equal components at all other
frequencies due to noise. After a low-pass Butterworth filter is applied, the
frequency components above the cutoff frequency are attenuated by \mbox{20
dB/dec} per pole (or filter order). We can see in Figure \ref{fig:1b} that
attenuation of the noise begins around \mbox{25 Hz}, the cutoff frequency of
the applied filter, at a rate of \mbox{60 dB/dec} as the filter is third order.

% 1c
\subsection{Effect of Observation Time}
\begin{figure}
    \begin{center}
        \includegraphics[width=\textwidth]{question1c.eps}
    \end{center}
    \caption{Magnitude spectrum of signal $y(t)$ observed for different lengths
    of time}
    \label{fig:1c}
\end{figure}
A longer observation time $T$ of a time signal results in higher resolution in
the Frequency domain. In figure \ref{fig:1c}, a signal is sampled twice, once
for 10 seconds at \mbox{1000 Hz} and once for 5 seconds at \mbox{1000 Hz}, and
the resulting Fourier transforms are plotted. Note, the Fourier transforms in
figure \ref{fig:1c} have not been scaled by the length of the input signal or
any other value so that the magnitudes of the Fourier coefficients can be
compared in this case with equal sampling rate but differing observation time.

As the sampling frequency is the same, both Fourier transforms have frequency
components up to \mbox{1000 Hz}, but the signal with observation time $T = 10
s$, $y_{1000}$, has a frequency resolution of \mbox{0.1 Hz} while the signal
with observation time $T = 5 s$, $y_{1000, 5}$, has a frequency resolution of
\mbox{0.2 Hz}. While the components cannot be seen every 0.1 and \mbox{0.2 Hz}
in figure \ref{fig:1c}, the effect of a different frequency resolutions with a
noisy signal are visible. Since noise is present at all frequencies in
equivalent amounts, a lower sampling rate will result in larger frequency
components due to noise as the each FFT bin encompasses a large range of
frequencies. This can be seen by the higher noise floor in signal $y_{1000, 5}$
over signal $y_{1000}$.

\section{Open Loop Identification}

% 2a
\subsection{Theoretical Input/Output Spectrum}
Given that input signal $u(t)$ is white noise, the theoretical input spectrum
$U(f)$ is simply a constant power spectral density with value $\sigma_u^2$.
Then the theoretical output spectrum $Y(f)$ is the system frequency reponse
$H(f)$ scaled by $\sigma_u^2$. This can be seen by representing the output
$y(t)$ as a the convolution of input $u(t)$ and impulse response function
$h(t)$ and then transforming to the frequency domain:
\begin{align*}
    \mathcal{F}(y(t)) &= \mathcal{F}(\sum_{k=0}^{T-1} h(k)u(t - k)]) \\
    Y(f) &= U(f) H(f) \\
    Y(f) &= \sigma_u^2 H(f)
\end{align*}

% 2b
\subsection{Estimation of system H}
Starting with:
\begin{equation*}
    Y(f) = H(f)U(f) + N(f)
\end{equation*}
we can multiply by $U^*(f)$ to obtain the spectra:
\begin{align*}
    U^*(f)Y(f) &= U^*(f)H(f)U(f) + U^*(f)N(f) \\
    S_{yu}(f) &= S_{uu}(f)H(f) + S_{nu}(f)
\end{align*}
Rearraging terms:
\begin{equation}
    H(f) = \frac{S_{yu}(f)}{S_{uu}(f)}  - \frac{S_{nu}(f)}{S_{uu}(f)}
    \label{eq:H}
\end{equation}
Knowing that cross spectrum of independent white noise signals is zero:
\begin{align*}
    0 &= \phi_{nu}(\tau) \\
      &= \mathcal{F}(\phi_{nu}(\tau)) \\
      &= S_{nu}(f)
\end{align*}
Thus we can find the estimate $\hat{H}(f)$ with the estimates of power spectra:
\begin{equation}
    \hat{H}(f) = \frac{\hat{S}_{yu}(f)}{\hat{S}_{uu}(f)}
    \label{eq:H_est}
\end{equation}


% 2c
\subsection{Magnitude, phase, and coherence of estimate $\hat{H}$}
\begin{figure}
    \begin{center}
        \includegraphics[width=\textwidth]{question2c.eps}
    \end{center}
    \caption{Open-loop spectral estimation of H}
    \label{fig:2c}
\end{figure}
Figure \ref{fig:2c} displays the magnitude, phase, and coherence of $\hat{H}$.
It is clear that noise is distorting the magnitude and phase spectral density
plots. The coherence plot shows a constant value of 1 due to lack of averaging.

% 2d
\subsection{Effects of Welch averging on estimate $\hat{H}$}
\begin{figure}
    \begin{center}
        \includegraphics[width=\textwidth]{question2d.eps}
    \end{center}
    \caption{Open-loop spectral estimation of H with Welch averaging with
    varying number of segments}
    \label{fig:2d}
\end{figure}
Figure \ref{fig:2d} shows the effect of Welch averaging with different numbers
of segments. By splitting the time signal into multiple segments, computing the
FFT of each segment, and then averaging the resulting the power spectra, the
variance in the Fourier transform is reduced. At the expense of loss of
resolution due to Fourier transforms with lower observation times, random error
due to noise can be reduced, and a better estimate for coherence can be
obtained.

% 2e
\subsection{Motivation for averaging methods}
Averaging reduces variance in the estimated power spectrum at the expense
of reduced frequency resolution. Averaging also allows estimation of components
at higher frequencies as it reduces the effect of noise. Futhermore, the
coherence estimator is always biased and converges to the correct value with
averaging. Without averaging, the coherence equals 1 due to its definition.

% 2f
\subsection{High variance in input signal $u(t)$}
\begin{figure}
    \begin{center}
        \includegraphics[width=\textwidth]{question2f.eps}
    \end{center}
    \caption{Open-loop spectral estimation of H with different noise variance}
    \label{fig:2f}
\end{figure}
By increasing the variance of the input signal $u(t)$ by a factor of $100$
($\sigma_u^2 = 50$), we see in figure \ref{fig:2f} that the error between the
estimated and true frequency response has been reduced. Comparing the coherence
in figure \ref{fig:2f} to figure \ref{fig:2d} shows better coherence for
variance $\sigma_u^2 = 50$ (estimates in figure \ref{fig:2f} use Welch
averaging with $N = 10$). As seen in equation (\ref{eq:H}), $H$ is proportional
to the difference in $S_{yu}$ and $S_{nu}$. While $S_{nu} = 0$ theoretically,
this value will be non-zero for a single realization and we have the
expression:
\begin{align}
    \hat{H}(f) &= \frac{\hat{S}_{yu}(f)}{\hat{S}_{uu}(f)}  -
    \frac{\hat{S}_{nu}(f)}{\hat{S}_{uu}(f)} \notag\\
    \hat{H}(f) &= \frac{\hat{H}(f)\hat{S}_{uu}(f)}{\hat{S}_{uu}(f)}  -
    \frac{\hat{S}_{nu}(f)}{\hat{S}_{uu}(f)}
    \label{eq:H_est_n}
\end{align}
where we have substituted in equation (\ref{eq:H_est}). As $S_{uu} =
\sigma_u^2$, increasing $\sigma_u^2$ will increase the contribution of the
first term relative to the second term of equation (\ref{eq:H_est_n}),
resulting in a reduction of error in $\hat{H}(f)$ due to additive output noise
$n(t)$.

While a theoretically ideal input power would be very large (or infinite), this
is achiveable in system identification experiments. Due to physical
limitations, extremely large power inputs cannot always be supplied, and can
also drive a system outside its linear operating region or damage the system
itself. Furthermore, all experiments have finite observation time. Increasing
the variance of the input signal requires the observation time also be extended
such that the estimated input power spectrum approaches a constant value.


% 2g
\subsection{High variance in additive output noise signal $n(t)$}
While theoretically high variance in the additive ouput noise $n(t)$ should not
affect the frequency response $H(f)$ as the cross-spectrum $S_{nu} = 0$, this
will not be the case for in data collected from experiments. By increasing the
variance $\sigma_n^2$ of signal $n(t)$, the estimate of cross-spectrum
$\hat{S}_{nu}$ is increased. Referring to equation (\ref{eq:H_est_n}), this
will increase the second term and increase error in the estimate $\hat{H}(f)$.
This can be seen in figure \ref{fig:2f} where the noise variance has been
increased by a factor of $100$ ($\sigma_n^2 = 10$) and an estimate $\hat{H}(f)$
is obtained using Welch averaging with $N = 10$.

\end{document}
