\documentclass[times,12pt,reqno]{amsart}
% See geometry.pdf to learn the layout options. There are lots.
\usepackage{geometry}
\geometry{left = 1in, right = 1in, top = 1in, bottom = 1in}
% Activate to begin paragraphs with an empty line rather than an indent
\usepackage[parfill]{parskip}
\usepackage{courier}
\usepackage{multirow}
\usepackage{fancyhdr}
\usepackage{float,amsmath,graphicx,color,framed,subfiles}
\usepackage{amssymb,subcaption,textcomp,listings}
\usepackage{booktabs,epstopdf,caption,units}
\usepackage[table]{xcolor}

\lstset{
language=Matlab,
basicstyle=\footnotesize\ttfamily,
showspaces=false,
showstringspaces=false,
showtabs=false,
frame=single,
tabsize=2,
captionpos=b,
breaklines=false,
breakatwhitespace=false,
numbers=left,
numbersep=4pt,
}

\pagestyle{fancy}
\fancyhead{}
\fancyhead[R]{Assignment 1 - Lee, Sharma}

\title{WB2301-5 System Identification and Parameter Estimation \\
Assignment 1 - Introduction to System Identification}
\author{Oliver Lee\\Nikhil Sharma}
\graphicspath{ {images/} }

\begin{document}
\maketitle

\section{Question 1: Signal Analysis in the Time Domain}

% 1a
\subsection{}

Cross-covariance of signals $u$, $y$ are plotted in figure \ref{fig:q1a}. Signals
$u$, $y$ are observed for 5 seconds at a sampling frequency of 100 Hz.
\begin{figure}[H]
    \begin{center}
        \includegraphics[width=\textwidth]{question1a.eps}
    \end{center}
    \caption{Cross-Covariances of signals $u$, $y$}
    \label{fig:q1a}
\end{figure}

% 1b
\subsection{}
The variable on the x-axis is time delay $\tau$, which is the time shift
between the two signals. The y-axis is the covariance between the time shifted
signals $u$, $y$.

For $C_{uy}$, $u$ is the output, $y$ is the input. For $C_{yu}$, $y$ is the
output, $u$ is the input. Since the two cross-covariances simply have the input
and output switched, the two cross-covariances are mirrored about $\tau = 0$.
This can be expressed as
\begin{equation*}
    C_{uy}(\tau) = C_{yu}(-\tau)
\end{equation*}

% 1c
\subsection{}
Given that signal $u$ is random white noise and that each random variable is
independent of all previous random variables, theoretically
\begin{align*}
    C_{uu}(\tau) &= 0, \: \forall \: \tau \neq 0 \\
    C_{uu}(0) &= \sigma_u^2
\end{align*}

For signals $n$, $u$ that are both white noise, but independent processes
\begin{align*}
    C_{nu}(\tau) &= 0, \: \forall \: \tau
\end{align*}

% 1d
\subsection{}
As lag increases, the estimate suffers from large variances as the value
is computed with fewer data points. As the lag becomes large, the leading
$\frac{1}{N - \tau}$ factor becomes large. To avoid this, a biased estimator
can be used, which does not use lag in the scaling factor.

\newpage
% 1e
\subsection{}

The following function implements the cross-covariance of two time signals
using a frequency domain approach.

\lstset{caption=Cross-Covariance Function}
\lstinputlisting{../crosscov.m}

The crosscov() function uses similar input and output variables as xcov() and
the user should refer to that documentation for details on y, u, maxlag,
scaleopt, c, and lags.

Lines 4 - 15 handle input arguments. Lines 21 - 26 compute the
cross-correlation in the frequency domain. Given that the cross spectrum is the
Fourier transform of the autocorrelation and that the cross spectrum is product
of the Fourier transforms of two signals, the implementation in MATLAB is
straight foward. Note that the cross-covariance is calculated by removing the
mean from each signal. Lines 28 - 32 will scale the cross-covariance terms such
that it is either biased or unbiased.

% 1f
\subsection{}
Calculating the argmax of the cross-correlation coefficients $K_{uu}(\tau)$,
$K_{yu}(\tau)$ gives
%\begin{align*}
%    argmax \: K_{uu}(\tau) &= 0 s \\
%    argmax \: K_{yu}(\tau) &= 0.1800 s
%\end{align*}

\begin{center}
    \begin{tabular} {|r|r|}
        \hline
        $argmax \: K_{uu}$ & $0 s$ \\
        $argmax \: K_{yu}$ & $0.1800 s$ \\
        \hline
    \end{tabular}
\end{center}

% 1g
\subsection{}
Calculating the sum of squares for the error for each covariance calculation
shows an extremely small difference between the functions xcov() and
crosscov()

\begin{center}
    \begin{tabular} {|r|r|r|}
        \hline
        \multicolumn{3}{|c|}{Sum of Squares Error} \\
        \hline
        \nonumber & ${biased}$ & ${unbiased}$ \\
        \hline
        $C_{uu}$ & $3.7656 \times 10^{-31}$ & $1.1013 \times 10^{-28}$ \\
        $C_{uy}$ & $7.7604 \times 10^{-32}$ & $4.3742 \times 10^{-29}$ \\
        $C_{yu}$ & $6.0888 \times 10^{-32}$ & $2.7834 \times 10^{-29}$ \\
        $C_{yy}$ & $3.0425 \times 10^{-31}$ & $6.7404 \times 10^{-29}$ \\
        \hline
    \end{tabular}
\end{center}

% 1h
\subsection{}
The system impulse response and $C_{yu}(\tau)$ are plotted in figure
\ref{fig:q1h}. The impulse response of a system comprises of excitations of
the system at all frequencies in equal amounts. This can be approximated with a
white noise time signal as a white noise signal should have also have all
frequency components in equal amounts. We should expect that the
cross-covariance of the output $y$ with white noise to be proportional to the
impulse response.
\begin{figure}[H]
    \begin{center}
        \includegraphics[width=\textwidth]{question1h.eps}
    \end{center}
    \caption{System impulse response and $C_{yu}(\tau)$}
    \label{fig:q1h}
\end{figure}

% 1i
\subsection{}
The cross-covariance of signals $y$, $u$ for 100 seconds, $C_{yu}(\tau)$, has
been plotted along with cross-covariance of the same signals for 5 seconds,
$C_{yu}(\tau)$, and the system impulse response in figure \ref{fig:q1i}. Given
that the signal $u$ is longer, it should be closer to theoretical white noise,
that is, contain all frequency components at equal values. Thus, the
cross-covariance of the longer signals, $C_{yu100}$, is a better approximation
of the impulse response than $C_{yu}$.
\begin{figure}[H]
    \begin{center}
        \includegraphics[width=\textwidth]{question1i.eps}
    \end{center}
    \caption{System impulse response, $C_{yu}(\tau)$, $C_{yu100}(\tau)$}
    \label{fig:q1i}
\end{figure}

\newpage
\section{Question 2: Fourier transform and spectral densities}

% 2a
\subsection{}
The magnitude plot of $Y(f)$, shown in figure \ref{fig:q2a}, we see the gain of the
signal $y$ for different frequencies. If the system is assumed to be linear and
the input is a sinusoid, we can see the gain of the system for specific
frequencies. Note, magnitude of $Y(f)$ is mirrored about the nyquist frequency.
\begin{figure}[H]
    \begin{center}
        \includegraphics[width=\textwidth]{question2a.eps}
    \end{center}
    \caption{Magnitude and phase of Fourier transform of $y(t)$, $Y(f)$}
    \label{fig:q2a}
\end{figure}

% 2b
\subsection{}

The spectral density plots obtained using the direct approach, transforming the
signals into frequency domain and then deriving the spectral densities, are
shown in figure \ref{fig:q2b}.
\begin{figure}[H]
    \begin{center}
        \includegraphics[width=\textwidth]{question2b.eps}
    \end{center}
    \caption{Spectral densities $S_{uu}(f)$, $S_{yu}(f)$ (direct)}
    \label{fig:q2b}
\end{figure}

% 2c
\subsection{}

The spectral density plots obtained using the indirect approach, deriving the
covariance functions and then transforming to frequency domain, are shown in
figure \ref{fig:q2c}.
\begin{figure}[H]
    \begin{center}
        \includegraphics[width=\textwidth]{question2c.eps}
    \end{center}
    \caption{Spectral densities $S_{uu}(f)$, $S_{yu}(f)$ (indirect)}
    \label{fig:q2c}
\end{figure}


% 2d
\subsection{}

Using the direct and indirect approach result in differences in the calculation
of spectral densities. Due to high amount of variance associated with large
lags in the biased cross-covariances, windowing should be used before
calculating the FFT. However, lack of windowing and biased calculations of the
cross-covariances results in a higher effect of noise.

\newpage
\section{Question 3: Time Domain Models}

% 3a
\subsection{}
The response of the ARX model looks completely different than the response of
the system. The height of the peak around 1.5 Hz in the magnitude plot is not
as sharp in the ARX model as that of the true system. The phase doesn't match
at all. The changes in slope of the phase plot shows that the ARX model has
many more zeros and poles than the system. It's probably because the model
being used is wrong. The Bode plot of both model and system are presented in
figure \ref{fig:q3a}.
\begin{figure}[H]
    \begin{center}
        \includegraphics[width=\textwidth]{question3a.eps}
    \end{center}
    \caption{Bode Diagram of system and ARX model}
    \label{fig:q3a}
\end{figure}

% 3b
\subsection{}
Looking at the system, we see that we have a single input (white noise,
signal u) which we treat as a deterministic input as well as noise n, a
stochastic input. Immediately we can rule out models what do not incorporate
both a deterministic and stochastic input: Finite Impulse Response (FIR),
autoregressive (AR), and autoregressive moving average (ARMA). We are then left
with 4 models: Output Error (OE), autoregressive exogenous input (ARX),
autoregressive moving average exogenous input (ARMAX), and Box-Jenkins (BJ).
While all can be used, as each is a more generalized version of the previous
model, in the order presented, the simpliest one that can be applied is OE due
to lack of filtering of noise n.

% 3c
\subsection{}

Using MATLAB function compare(), the NRMSE fitness values for the different
model structures are
\begin{center}
    \begin{tabular} {|r|r|r|r|r|r|}
        \hline
        \multicolumn{6}{|c|}{NRMSE fitness for System Models} \\
        \hline
        \nonumber & ${True}$ & ${OE}$ & ${ARX}$ & ${ARMAX}$ & ${BJ}$ \\
        \hline
        $fit$ & $53.6265$ & $53.6518$ & $2.4514$ & $53.6480$ & $53.6560$ \\
        \hline
    \end{tabular}
\end{center}

We see that the fitness values are roughly the same for OE, ARMAX, and BJ model
structures as predicted. A Bode diagram of the OE model and the true system are
presented in figure \ref{fig:q3c}. It is clear that the Bode magnitude and
phase match closely up until the nyquist frequency.
\begin{figure}[H]
    \begin{center}
        \includegraphics[width=\textwidth]{question3c.eps}
    \end{center}
    \caption{Bode Diagram of system and OE model}
    \label{fig:q3c}
\end{figure}

% 3d
\subsection{}
If it is assumed that model numbers are the same and the model structure is
given:
\begin{enumerate}
    \item Start with model order = 0;
    \item Increment model order.
    \item Estimate model using input data and model order.
    \item Compare it to data.
\end{enumerate}
If the fit value is within a specified margin of 100 or
if the fit value stops increasing after a given amount of time, then stop as
the model order has been found. Otherwise, go back to step 1.

If model orders are not the same ($ n_a \neq n_b $, etc.), and the model
structure is unknown, I would use a Genetic Algorithm with the candidate
solutions being composed of the model structures and model orders and the NRMSE
as the fitness function that is used to evaluate the candidate solutions.
During selction of subsequent generations, a small percentage (~10\%) of the
most fit candidates would be selected for retention, and the rest of the
candidates would be generated randomly or by generated from combining fit
solutions from the previous generation. Termination conditions would be the
same as above.

\end{document}
